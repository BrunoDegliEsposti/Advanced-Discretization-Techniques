\documentclass[a4paper]{article}
\usepackage[T1]{fontenc}
\usepackage[utf8]{inputenc}
\usepackage[english]{babel}
\usepackage{amsmath}
\usepackage{amsfonts}
\usepackage{amssymb}
\usepackage{enumitem}
\usepackage{resizegather} \addtolength{\jot}{4pt}
\usepackage{microtype}

\renewcommand{\vec}[1]{\mathbf{#1}}
\DeclareMathOperator*{\argmin}{\arg\!\min}
\DeclareMathOperator*{\argmax}{\arg\!\max}
\DeclareMathOperator{\sgn}{sgn}
\DeclareMathOperator{\diver}{div}
\DeclareMathOperator{\supp}{supp}
\newcommand{\dx}{\, dx \,}
\newcommand{\dy}{\, dy \,}
\newcommand{\dt}{\, dt \,}
\newcommand{\ds}{\, ds \,}
\newcommand{\dxdy}{\, dx \:\! dy \,}
\newcommand{\dvecx}{\, d\vec{x} \,}
\newcommand{\dsigma}{\, d\sigma \,}
\newcommand{\area}[1]{\left\lvert #1 \right\rvert}
\newcommand{\abs}[1]{\left\lvert #1 \right\rvert}
\newcommand{\seminorm}[1]{\left\lvert #1 \right\rvert}
\newcommand{\norm}[1]{\left\lVert #1 \right\rVert}
\newcommand{\dpair}[1]{\left\langle #1 \right\rangle}
\newcommand{\R}{\mathbb{R}}
\newcommand{\N}{\mathbb{N}}

\title{\huge{Advanced Discretization Techniques \\ Homework 10}}
\author{\Large{Bruno Degli Esposti, Xingyu Xu}}
\date{January 7th - January 14th, 2020}

\begin{document}

\maketitle

\section*{Exercise 19: Estimates for parabolic equations}
\begin{enumerate}[label=\textbf{\alph*)},leftmargin=*]
\item We first prove a stronger, differential version of the inequality.
	If $\norm{u(t)}_0 = 0$, then we're already done.
	Otherwise, by the usual theorem for differentiation
	under the (Lebesgue) integral sign,
	\[
	\frac{d}{dt} \norm{u(t)}_0
	= \frac{d}{dt} \dpair{u(t),u(t)}^{1/2}
	= \frac{2 \dpair{u'(t),u(t)}_0} {2 \norm{u(t)}_0}
	= \frac{\dpair{u'(t),u(t)}_0} {\norm{u(t)}_0}.
	\]
	For $v = u(t)$, the last numerator is equal to the first integral in the
	weak formulation of the boundary value problem, so we get
	\[
	\frac{\dpair{u'(t),u(t)}_0} {\norm{u(t)}_0}
	= \frac{\int_\Omega u'(t) u(t) \dx} {\norm{u(t)}_0}
	= \frac{\int_\Omega f(t) u(t) \dx - a(u(t),u(t))} {\norm{u(t)}_0}.
	\]
	By the positive definiteness of the bilinear form $a(\cdot,\cdot)$
	and Cauchy-Schwarz's inequality, we can conclude that
	\[
	\frac{\int_\Omega f(t) u(t) \dx - a(u(t),u(t))} {\norm{u(t)}_0}
	\leq \frac{\int_\Omega f(t) u(t) \dx} {\norm{u(t)}_0}
	\leq \frac{\norm{f(t)}_0 \norm{u(t)}_0} {\norm{u(t)}_0}
	= \norm{f(t)}_0.
	\]
	Now we can integrate the differential version of the inequality over time:
	\begin{align*}
	\int_0^s \frac{d}{dt} \norm{u(t)}_0 \dt
&	\leq \int_0^s \norm{f(t)}_0 \dt \\
	\norm{u(s)}_0
&	\leq \norm{u(0)}_0 + \int_0^s \norm{f(t)}_0 \dt.
	\end{align*}
\item Again, we first prove a stronger, differential version of the inequality.
	By Poincaré's inequality in $H_0^1(\Omega)$, there exists a constant $C_1 > 0$
	independent of $t$ such that
	\[
	\norm{u(t)}_0^2 \leq C_1^2 \seminorm{u(t)}_1^2
	\]
	holds for all $u(t) \in H^1(0,T;V)$.	Then, taking $v = 2u(t)$, it follows that
	\begin{align*}
	\frac{d}{dt} \norm{u(t)}_0^2
&	= \dpair{u'(t),2u(t)}_0 \\
&	= \dpair{f(t),2u(t)}_0 - a(u(t),2u(t)) \\
&	= \int_\Omega 2f(t)u(t) \dx - \seminorm{u(t)}_1^2 - \seminorm{u(t)}_1^2 \\
&	\leq \int_\Omega 2f(t)u(t) \dx - C_1^{-2} \norm{u(t)}_0^2 - \seminorm{u(t)}_1^2 \\
&	= \int_\Omega C_1^2 f(t)^2 \dx
		- \int_\Omega \big( C_1 f(t) - C_1^{-1} u(t) \big)^2 \dx
		- \seminorm{u(t)}_1^2 \\
&	\leq C_1^2 \norm{f(t)}_0^2 - \seminorm{u(t)}_1^2.
	\end{align*}
	In the last steps, we've completed the square and discarded a negative term.
	Now we can integrate the differential version of the inequality over time:
	\begin{gather*}
	\int_0^s \frac{d}{dt} \norm{u(t)}_0^2 \dt
	\leq \int_0^s C_1^2 \norm{f(t)}_0^2 \dt - \int_0^s \seminorm{u(t)}_1^2 \dt \\
	\norm{u(s)}_0^2 - \norm{u(0)}_0^2
	\leq C_1^2 \int_0^s \norm{f(t)}_0^2 \dt - \int_0^s \seminorm{u(t)}_1^2 \dt \\
	\norm{u(s)}_0^2 + \int_0^s \seminorm{u(t)}_1^2 \dt
	\leq \norm{u(0)}_0^2 + C_1^2 \int_0^s \norm{f(t)}_0^2 \dt
	\end{gather*}
\item The proof is very similar to the one of point b).
	The only differences are that this time we choose $v = 2u'(t)$
	and use the symmetry of $a(\cdot,\cdot)$:
	\begin{align*}
	\frac{d}{dt} \seminorm{u(t)}_1^2
&	= \frac{d}{dt} a(u(t),u(t))
	= a(u(t),2u'(t)) \\
&	= \int_\Omega 2f(t)u'(t) \dx - \int_\Omega u'(t)^2 \dx - \int_\Omega u'(t)^2 \dx \\
&	= \int_\Omega f(t)^2 \dx - \int_\Omega (f(t)-u'(t))^2 \dx - \int_\Omega u'(t)^2 \dx \\
&	\leq \int_\Omega f(t)^2 \dx - \int_\Omega u'(t)^2 \dx
	= \norm{f(t)}_0^2 - \norm{u'(t)}_0^2
	\end{align*}
	Now we can integrate the differential version of the inequality over time:
	\begin{align*}
	\int_0^s \frac{d}{dt} \seminorm{u(t)}_1^2 \dt
	\leq \int_0^s \norm{f(t)}_0^2 \dt - \int_0^s \norm{u'(t)}_0^2 \dt \\
	\seminorm{u(s)}_1^2 - \seminorm{u(0)}_1^2
	\leq \int_0^s \norm{f(t)}_0^2 \dt - \int_0^s \norm{u'(t)}_0^2 \dt \\
	\seminorm{u(s)}_1^2 + \int_0^s \norm{u'(t)}_0^2 \dt
	\leq \seminorm{u(0)}_1^2 + \int_0^s \norm{f(t)}_0^2 \dt
	\end{align*}
\item By Poincaré's inequality in $H_0^1(\Omega)$,
	there exists a constant $C_2 > 0$ independent of $t$ such that
	\[
	\norm{u(t)}_1^2 \leq C_2^2 \seminorm{u(t)}_1^2
	\]
	holds for all $u(t) \in H^1(0,T;V)$.
	Let $v \in V$ be the weak solution to the given Poisson problem.
	This $v$ can also be seen as a constant function in $H^1(0,T;V)$.
	Then, by taking the difference of the weak formulations,
	we immediately get that $u(t)-v$ is a weak solution to the
	corresponding homogenous parabolic equation ($f(t) \equiv 0$),
	and so the three previous estimates also apply to $u(t)-v$.
	In particular, if we sum the differential versions of inequalities b) and c),
	we get that
	\begin{align*}
	\frac{d}{dt} \norm{u(t)-v}_1^2
&	\leq -\seminorm{u(t)-v}_1^2 - \norm{\partial_t(u(t)-v)}_0^2 \\
&	\leq -\seminorm{u(t)-v}_1^2
	\leq -C_2^{-2} \norm{u(t)-v}_1^2.
	\end{align*}
	By Grönwall's lemma (in differential form), it follows that
	\[
	\norm{u(t)-v}_1^2
	\leq \norm{u(0)-v}_1^2 e^{\int_0^t -C_2^{-2} \ds}
	= \norm{u(0)-v}_1^2 e^{-tC_2^{-2}} \ ,
	\]
	and this implies
	\[
	\lim_{t\to\infty}\|u(t)-v\|_1 = 0,
	\]
	as required. $\square$
\end{enumerate}
\end{document}





























