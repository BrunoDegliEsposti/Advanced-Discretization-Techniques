\documentclass[a4paper]{article}
\usepackage[T1]{fontenc}
\usepackage[utf8]{inputenc}
\usepackage[english]{babel}
\usepackage[fleqn]{amsmath}
\usepackage{amsthm}
\usepackage{amssymb}
\usepackage{enumitem}
\usepackage{resizegather} \addtolength{\jot}{4pt}
\usepackage{microtype}

\renewcommand{\vec}[1]{\mathbf{#1}}
\DeclareMathOperator*{\argmin}{\arg\!\min}
\DeclareMathOperator*{\argmax}{\arg\!\max}
\DeclareMathOperator{\sgn}{sgn}
\DeclareMathOperator{\diver}{div}
\DeclareMathOperator{\supp}{supp}
\newcommand{\dx}{\, dx \,}
\newcommand{\dy}{\, dy \,}
\newcommand{\dt}{\, dt \,}
\newcommand{\ds}{\, ds \,}
\newcommand{\dxdy}{\, dx \:\! dy \,}
\newcommand{\dvecx}{\, d\vec{x} \,}
\newcommand{\dsigma}{\, d\sigma \,}
\newcommand{\area}[1]{\left\lvert #1 \right\rvert}
\newcommand{\abs}[1]{\left\lvert #1 \right\rvert}
\newcommand{\seminorm}[1]{\left\lvert #1 \right\rvert}
\newcommand{\norm}[1]{\left\lVert #1 \right\rVert}
\newcommand{\dpair}[1]{\left\langle #1 \right\rangle}
\newcommand{\R}{\mathbb{R}}
\newcommand{\N}{\mathbb{N}}
\newcommand{\C}{\mathbb{C}}
\renewcommand{\Re}{\operatorname{Re}}
\renewcommand{\Im}{\operatorname{Im}}

\title{\huge{Advanced Discretization Techniques \\ Homework 12}}
\author{\Large{Bruno Degli Esposti, Xingyu Xu}}
\date{January 21st - January 28th, 2020}

\begin{document}

\maketitle

\section*{Exercise 22: Fully discrete error estimate}
\begin{enumerate}[label=\textbf{\alph*)},leftmargin=*]
\item For any fixed time $t$, consider the following elliptic boundary
	value problem in weak form:
	\[
	\text{Find } w \in V \text{ such that } a(w,v) = a(u(t),v) \text{ for each } v \in V.
	\]
	Of course, $w = u(t)$ is a solution, and it's actually unique
	because of Lax-Milgram's lemma. On the other hand, the finite
	element solution of the problem will be some $w_h \in V_h$
	such that $a(w_h,v_h) = a(u(t),v_h)$ for each $v_h \in V_h$,
	and we must then have $w_h = R_h u(t)$ by the definition of $R_h$.
	Now, the Aubin/Nitsche theorem (3.37 in the book by Knabner and Angermann,
	inequality (3)) gives us exactly what we need to prove:
	\[
	\norm{u(t) - R_h u(t)}_0 \leq C h^2 \seminorm{u(t)}_2
	\]
	However, we still need to check that the hypothesis of the theorem hold.
	In particular, we need to check that the adjoint boundary value problem
	is regular. As $a(\cdot,\cdot)$ is symmetric, the adjoint boundary value problem
	is the same as the original one. Regularity follows from
	Lax-Milgram's lemma and the elliptic regularity estimate
	\[
	\seminorm{u(t)}_2 \leq C \norm{a(u(t),\cdot)}.
	\]
	The proof of this estimate can be found in Evans' book on partial
	differential equations (see theorem 4, chapter 6.3, page 317).
\item By point a) and the fundamental theorem of calculus for Bochner integrals,
	it follows that
	\[
	\norm{u(t_n) - R_h u(t_n)}_0
	\leq C h^2 \seminorm{u(t_n)}_2
	= C h^2 \seminorm{u_0 + \int_0^{t_n} \partial_t u(t) \dt}_2.
	\]
	Then, by the triangle inequality (once for sums, once for integrals),
	\begin{align*}
	C h^2 \seminorm{u_0 + \int_0^{t_n} \partial_t u(t) \dt}_2
&	\leq Ch^2\left(|u_0|_2+\left|\int^{t_n}_0\partial_tu(t)\dt\right|_2\right) \\
&	\leq Ch^2 \left( \seminorm{u_0}_2
		+ \int_0^{t_n} \seminorm{\partial_t u(t)}_2 \dt \right) \\
&	= Ch^2\left(\|\partial_{xx}u_0\|_0+\int^{t_n}_0\|\partial_{t,xx}u(t)\|_0\dt\right).
	\end{align*}
\item By the definitions of $\theta_n$, implicit Euler scheme and Ritz-projection,
	we can prove the following chain of equalities:
	\begin{gather*}
	\left(\frac{\theta_n-\theta_{n-1}}{\tau_{n-1}},v_h\right)_0
	  + a(\theta_n,v_h) \\
	= \left(\frac{R_h u(t_n)-U^n-R_h u(t_{n-1})+U^{n-1}}{\tau_{n-1}},v_h\right)_0
	  + a(R_h u(t_n)-U^n,v_h) \\
	\begin{split} = \left(\frac{R_h u(t_n)-R_h u(t_{n-1})}{\tau_{n-1}},v_h\right)_0
	  &+ a(R_h u(t_n),v_h) \\[-4pt]
	  &- \left(\frac{U^n-U^{n-1}}{\tau_{n-1}},v_h\right)_0
	  - a(U^n,v_h) \end{split} \\
	= \left(\frac{R_h u(t_n)-R_h u(t_{n-1})}{\tau_{n-1}},v_h\right)_0
	  + a(R_h u(t_n),v_h) - (f^n,v_h)_0 \\
	= \left(\frac{R_h u(t_n)-R_h u(t_{n-1})}{\tau_{n-1}},v_h\right)_0
	  + a(u(t_n),v_h) - (f^n,v_h)_0.
	\end{gather*}
	Since $u(t)$ solves the weak formulation of the problem,
	we can choose $v_h$ as a test function and get that
	$(\partial_t u(t_n), v_h)_0 + a(u(t_n),v_h) = (f^n,v_h)_0$. Then,
	\begin{gather*}
	\left(\frac{R_h u(t_n)-R_h u(t_{n-1})}{\tau_{n-1}},v_h\right)_0
		+ a(u(t_n),v_h) - (f^n,v_h)_0 \\
	= \left(\frac{R_h u(t_n)-R_h u(t_{n-1})}{\tau_{n-1}},v_h\right)_0
		- (\partial_t u(t_n), v_h)_0.
	\end{gather*}
\item In point c), we've essentially proved that $\theta_i$ satisfies
	an implicit Euler scheme with right-hand side $-\omega_i^1-\omega_i^2$:
	\begin{gather*}
	\frac{R_h u(t_i)-R_h u(t_{i-1})}{\tau_{i-1}} - \partial_t u(t_i) \\
	= \frac{R_h u(t_i) -u(t_i) +u(t_i) -u(t_{i-1}) +u(t_{i-1}) -R_h u(t_{i-1})}
		{\tau_{i-1}} - \partial_t u(t_i) \\
	= - \left( \partial_t u(t_i) - \frac{u(t_i)-u(t_{i-1})}{\tau_{i-1}} \right)
	  - \left( \frac{u(t_i) - R_h u(t_i)}{\tau_{i-1}}
	         - \frac{u(t_{i-1}) - R_h u(t_{i-1})}{\tau_{i-1}} \right).
	\end{gather*}
	Therefore, the fully discrete estimate from Exercise 21 tells us that
	\[
	\|\theta_n\|_0-\|\theta_0\|_0\leq\sum^{n}_{i=1}\tau_{i-1}\|-\omega^1_i-\omega^2_i\|_0,
	\]
	and then it follows by the triangle inequality and the definition of $\tau$ that
	\[
	\norm{\theta_n}_0
	\leq \norm{\theta_0}_0 + \tau\sum_{i=1}^n \norm{\omega_i^1 + \omega_i^2}_0
	\leq \norm{\theta_0}_0 + \tau\sum_{i=1}^n \norm{\omega_i^1}_0
		+ \tau\sum_{i=1}^n \norm{\omega_i^2}_0.
	\]
\item Taylor's formula with remainder in integral form gives us that
	\[
	u(t_{i-1})
	= u(t_i-\tau_{i-1})
	= u(t_i) + \partial_t u(t_i)(-\tau_{i-1})
		+ \int_{t_{i-1}}^{t_i} \partial_{tt} u(t) (t-t_{i-1}) \dt,
	\]
	therefore
	\begin{align*}
	\norm{\omega_i^1}_0
&	= \norm{\partial_t u(t_i)-\frac{u(t_i)-u(t_{i-1})}{\tau_{i-1}}}_0 \\
&	= \norm{\int_{t_{i-1}}^{t_i} \partial_{tt} u(t) \frac{t-t_{i-1}}{\tau_{i-1}} \dt}_0
	\leq \int_{t_{i-1}}^{t_i}
		\norm{\partial_{tt} u(t) \frac{t-t_{i-1}}{\tau_{i-1}}}_0 \dt \\
&	\leq \int_{t_{i-1}}^{t_i} \norm{\partial_{tt} u(t)}_0
		\frac{t_i-t_{i-1}}{\tau_{i-1}} \dt
	= \int_{t_{i-1}}^{t_i} \norm{\partial_{tt} u(t)}_0 \dt. \\
	\tau \sum_{i=1}^n \norm{\omega_i^1}_0
&	= \tau \sum_{i=1}^n \int_{t_{i-1}}^{t_i} \norm{\partial_{tt} u(t)}_0 \dt
	= \tau \int_0^{t_n} \norm{\partial_{tt} u(t)}_0 \dt.
	\end{align*}
\item By the fundamental theorem of calculus and point a) (which can be proved for
	$\partial_t u(t)$ as well, since $\partial_t u(t) \in H^2(0,1)$), we have that
	\begin{align*}
	\|\omega^2_i\|_0
&	= \norm{ \int^{t_i}_{t_{i-1}} \partial_t
		\left( \frac{u(t)-R_h(u(t))}{\tau_{i-1}} \right) \dt}_0 \\
&	= \norm{ \int^{t_i}_{t_{i-1}} 
		\left( \frac{\partial_t u(t)-R_h(\partial_t u(t))}{\tau_{i-1}} \right) \dt}_0 \\
&	\leq \int^{t_i}_{t_{i-1}} 
		 \frac{\norm{\partial_t u(t)-R_h(\partial_t u(t))}_0} {\tau_{i-1}} \dt \\
&	\leq\int^{t_i}_{t_{i-1}} \frac{Ch^2|\partial_t u(t)|_2}{\tau_{i-1}} \dt
	=\int^{t_i}_{t_{i-1}}\frac{Ch^2}{\tau_{i-1}}\|\partial_{t,xx}u(t)\|_0 \dt.
	\end{align*}
	If we now sum both sides for $i = 1,\dots,n$ and multiply them by $\tau$,
	we end up with
	\[
	\tau\sum^n_{i=1}\|\omega^2_i\|_0
	\leq\frac{\tau}{\min_{i=0,\dots,N-1}\tau_i}
		Ch^2\int^{t_n}_0\|\partial_{t,xx}u(t)\|_0 \dt
	\leq \tilde{C} h^2\int^{t_n}_0\|\partial_{t,xx}u(t)\|_0 \dt
	\]
	We've made the assumption that the time discretization is quasi-uniform.
\item By the triangle inequality and the definition of $\theta_n$,
	\[
	\|u(t_n)-U^n\|_0
	=\|u(t_n)-R_hu(t_n)+\theta_n\|_0\leq\|u(t_n)-R_hu(t_n)\|_0+\|\theta_n\|_0.
	\]
	Then, by points b) and d), it follows that
	\begin{multline*}
	\|u(t_n)-R_hu(t_n)\|_0+\|\theta_n\|_0
	\leq Ch^2\left(\|\partial_{xx}u_0\|_0+\int^{t_n}_0\|\partial_{t,xx}u(t)\|_0 \dt\right) \\
	+\norm{\theta_0}_0
	+\tau\sum^n_{i=1}\|\omega^1_i\|_0+\tau\sum^n_{i=1}\|\omega^2_i\|_0.
	\end{multline*}
	The term $\norm{\theta_0}_0$ vanishes because $R_h u(0) - U^0 = 0$ by assumption,
	and finally we can conclude the proof using the results from points e) and f):
	\begin{gather*}
	\|u(t_n)-U^n\|_0 \\
	\leq Ch^2\left(\|\partial_{xx}u_0\|_0
			+2\int^{t_n}_0\|\partial_{t,xx}u(t)\|_0dt\right)
			+\tau\int^{t_n}_0\|\partial_{tt}u(t)\|_0 \dt \\
	\leq 2Ch^2\left(\|\partial_{xx}u_0\|_0+\int^{t_n}_0\|\partial_{t,xx}u(t)\|_0dt\right)
		+\tau\int^{t_n}_0\|\partial_{tt}u(t)\|_0 \dt. \quad \square
	\end{gather*}
\end{enumerate}
\end{document}













