\documentclass[a4paper]{article}
\usepackage[T1]{fontenc}
\usepackage[utf8]{inputenc}
\usepackage[english]{babel}
\usepackage{amsmath}
\usepackage{amsfonts}
\usepackage{amssymb}
\usepackage{resizegather} \addtolength{\jot}{4pt}
\usepackage{microtype}

\renewcommand{\vec}[1]{\mathbf{#1}}
\DeclareMathOperator*{\argmin}{\arg\!\min}
\DeclareMathOperator*{\argmax}{\arg\!\max}
\DeclareMathOperator*{\esssup}{ess\sup}
\DeclareMathOperator{\sgn}{sgn}
\DeclareMathOperator{\diver}{div}
\DeclareMathOperator{\supp}{supp}
\newcommand{\dx}{\, dx \,}
\newcommand{\dy}{\, dy \,}
\newcommand{\dxy}{\, dx \:\! dy \,}
\newcommand{\dvecx}{\, d\vec{x} \,}
\newcommand{\dsigma}{\, d\sigma \,}
\newcommand{\area}[1]{\left\lvert #1 \right\rvert}
\newcommand{\abs}[1]{\left\lvert #1 \right\rvert}
\newcommand{\seminorm}[1]{\left\lvert #1 \right\rvert}
\newcommand{\norm}[1]{\left\lVert #1 \right\rVert}
\newcommand{\dpair}[1]{\left\langle #1 \right\rangle}
\newcommand{\R}{\mathbb{R}}
\newcommand{\N}{\mathbb{N}}
\newcommand{\Pone}{\mathbb{P}_1}

\title{\huge{Advanced Discretization Techniques \\ Homework 7}}
\author{\Large{Bruno Degli Esposti, Xingyu Xu}}
\date{December 3rd - December 10th, 2019}

\begin{document}

\maketitle

\section*{Exercise 15: Uzawa algorithm - a proof of convergence}
\begin{description}
\item[a)] The saddle point problem can be written in the following operator form:
	\begin{align*}
	Au + B^\ast p
&	= f \\
	Bu
&	= g.
	\end{align*}
	$A$ is invertible, because it's positive definite (by assumption).
	Hence, we can multiply the first equation by $BA^{-1}$ and
	substitute $g$ for $Bu$ to get that
	\begin{gather*}
	Bu + BA^{-1}B^{\ast}p = BA^{-1} f \\
	g + BA^{-1}B^{\ast}p = BA^{-1} f \\
	Sp = BA^{-1} f - g.
	\end{gather*}
\item[b)] Given a linear system $Lx=r$, the $\omega$-relaxated Richardson method
	is defined as the following iterative scheme:
	\[
	x^{k+1} = x^k + \omega(r-Lx^k).
	\]
	Uzawa's algorithm can be written in this form by substituting line 4 into line 6
	(and eliminating $u^{k+1}$ in the process):
	\begin{align*}
	p^{k+1}-p^k
&	= \omega(Bu^{k+1}-g)\\
&	= \omega(B(A^{-1}f-A^{-1}B^\ast p^k)-g)\\
&	= \omega(BA^{-1}f-g-BA^{-1}B^\ast p^k)\\
&	= \omega(BA^{-1}f-g-Sp^k),
	\end{align*}
	As required, $L = S$ and $r = BA^{-1}f-g$.
\item[c)] By the spectral theorem, $A^{1/2}$ is a well-defined, real,
	s.p.d.\ operator. Then
	\begin{gather*}
	\|v\|_A
	= \dpair{Av,v}_{H^1}^{1/2}
	= \dpair{A^{1/2}A^{1/2}v,v}^{1/2}
	= \dpair{A^{1/2}v,A^{1/2}v}^{1/2}
	= \norm{A^{1/2}v}_1 \\
	\norm{p}_S
	= \dpair{Sp,p}_{L^2}^{1/2}
	= \dpair{BA^{-1}B^*p,p}_{L^2}^{1/2}
	= \dpair{A^{-1/2}B^*p,A^{-1/2}B^*p}_{H^1}^{1/2}
	= \norm{A^{-1/2}B^*p}_1 \\
	\dpair{Bv,p}_{L^2}
	= \dpair{BA^{-1/2}A^{1/2}v,p}_{L^2}
	= \dpair{A^{1/2}v,A^{-1/2}B^*p}_{H^1}.
	\end{gather*}
	By the Cauchy-Schwarz inequality, it follows that
	\[
	\sup_{v \in M_h} \frac{\dpair{Bv,p}}{\norm{v}_A}
	= \sup_{v \in M_h} \frac{\dpair{A^{1/2}v,A^{-1/2}B^*p}}{\norm{A^{1/2}v}_1}
	\leq \norm{A^{-1/2}B^*p}_1
	= \norm{p}_S,
	\]
	and this inequality can be turned into an equality by choosing $v \in M_h$
	in the $\sup$ as $A^{-1}B^*p$, so that $A^{1/2}v = A^{-1/2}B^*p$.
\item[d)] In the Stokes case,
	\[
	a(u,v) = \int_\Omega Du : Dv \dx \qquad
	b(v,q) = -\int_\Omega q \diver(v) \dx.
	\]
	Let $R(q)$ be the Rayleigh quotient associated with $S$:
	\[
	R(q) = \frac{\dpair{Sq,q}}{\dpair{q,q}}.
	\]
	Since $S$ is s.p.d., it follows from the min-max theorem that
	\[
	\lambda_\text{min}(S) = \inf_{0 \neq q \in X_h} R(q)
	\quad \text{and} \quad
	\lambda_\text{max}(S) = \sup_{0 \neq q \in X_h} R(q).
	\]
	If we now combine the result from point c) with the inf-sup condition, we get that
	\begin{align*}
	\lambda_\text{min}(S)
&	= \inf_{0 \neq q \in X_h} \frac{\dpair{Sq,q}}{\dpair{q,q}}
	= \inf_{0 \neq q \in X_h} \frac{\norm{q}_S^2}{\norm{q}_0^2}
	%= \inf_{0 \neq q \in X_h} \sup_{0 \neq v \in M_h}
	= \inf_{q} \sup_{v} \frac{\dpair{Bv,q}^2}{\norm{v}_A^2 \norm{q}_0^2} \\
&	= \inf_{q} \sup_{v} \frac{\dpair{Bv,q}^2}{a(v,v) \norm{q}_0^2}
	= \inf_{q} \sup_{v} \frac{\dpair{Bv,q}^2}{\seminorm{v}_1^2 \norm{q}_0^2}
	= \beta^2,
	\end{align*}
	so the first half of the proof is complete. For the second half,
	we first need to show that $\norm{\diver(v)}_0^2 \leq \norm{v}_A^2$:
	\begin{align*}
	\norm{\diver(v)}_0^2
&	= \int_\Omega \left( \sum_{i=1}^d \partial_i v_i(x) \right)^2 \dx
	= \int_\Omega \sum_{i,j=1}^d \partial_i v_i(x) \partial_j v_j(x) \dx \\
&	= 0 + 0 + \int_\Omega \sum_{i,j=1}^d \partial_j v_i(x) \partial_i v_j(x) \dx \\
&	= \int_\Omega \sum_{i=1}^d (\partial_i v_i(x))^2 \dx
		+ \int_\Omega \sum_{1 \leq i < j \leq d}
		2 \partial_j v_i(x) \partial_i v_j(x) \dx \\
&	\leq \int_\Omega \sum_{i=1}^d (\partial_i v_i(x))^2 \dx
		+ \int_\Omega \sum_{1 \leq i < j \leq d}
		(\partial_j v_i(x))^2 + (\partial_i v_j(x))^2 \dx \\
&	= \int_\Omega \sum_{i,j=1}^d (\partial_i v_j(x))^2 \dx
	= a(v,v)
	= \norm{v}_A^2.
	\end{align*}
	We have used integration by parts twice (and, implicity, a density argument
	with $C_0^\infty(\Omega)$ functions), symmetry of second derivatives and the
	inequality $2ab \leq a^2 + b^2$. We can now conclude the second part of the proof:
	\begin{align*}
	\lambda_\text{max}(S)
&	= \sup_{0 \neq q \in X_h} \frac{\dpair{Sq,q}}{\dpair{q,q}}
	= \sup_{q} \sup_{v} \frac{b(v,q)^2}{\norm{v}_A^2 \norm{q}_0^2} \\
&	= \sup_{q} \sup_{v} \frac{
		\left( \int_\Omega q \diver(v) \dx \right)^2
		}{\norm{v}_A^2 \norm{q}_0^2}
	\leq \sup_{q} \sup_{v} \frac{
		\left( \int_\Omega q^2 \dx \right)
		\left( \int_\Omega \diver(v)^2 \dx \right)
		}{\norm{v}_A^2 \norm{q}_0^2} \\
&	= \sup_{q} \sup_{v} \frac{\norm{q}_0^2 \norm{\diver(v)}_0^2}
		{\norm{v}_A^2 \norm{q}_0^2}
	\leq \sup_{0 \neq v \in M_h} \frac{\norm{v}_A^2}{\norm{v}_A^2}
	= 1.
	\end{align*}
\item[e)] By induction on the error estimate for Richardson's method, we get that
	\[
	\norm{p-p^k}_0 \leq (\rho(I-S))^k \norm{p-p^0}_0.
	\]
	In point 4) we have shown that $\sigma(S) \subseteq [\beta^2,1]$,
	so $\sigma(I-S) \subseteq [0,1-\beta^2]$ and $\rho(I-S) \leq 1 - \beta^2$.
	This is enough to prove the first error estimate for Uzawa's algorithm.
	To prove the second one, consider the first equation of the saddle point problem
	and line 4 in the algorithm:
	\[
	Au + B^*p = f \qquad Au^{k+1} + B^*p^k = f.
	\]
	Subtracting one equation from the other gives
	\[
	A(u-u^{k+1})+B^\ast(p-p^k)=0,
	\]
	from which we can deduce that
	\begin{align*}
	\seminorm{u-u^{k+1}}_1^2
&	= \norm{u-u^{k+1}}_A^2
	= \dpair{A(u-u^{k+1}),u-u^{k+1}}_{H^1} \\
&	= \dpair{-B^*(p-p^k),-A^{-1}B^*(p-p^k)}_{H^1} \\
&	= \dpair{S(p-p^k),p-p^k}_{L^2}
	\leq \norm{S(p-p^k)}_0 \norm{p-p^k}_0 \\
&	\leq \norm{S}_2 \norm{p-p^k}_0^2
	\leq \rho(S) \norm{p-p^k}_0^2
	\leq (1-\beta^2)^{2k} \norm{p-p^0}_0^2.
	\end{align*}
	Now we just have to take the square root of both sides. $\square$
\end{description}

\section*{Exercise 16: Voronoi diagrams}
\begin{description}
\item[a)] Let $S_{ij} = \{x \in \R^2 \bigm| \abs{x-a_i} < \abs{x-a_j}\}$,
	so that $\tilde{\Omega}_i = \bigcap_{j \neq i} S_{ij}$.
	Geometric intuition suggests that the sets $S_{ij}$ are open half-planes,
	and that their boundaries $\partial S_{ij}$ are perpendicular bisectors
	of the line segments $a_i a_j$.
	To show that this is really the case, consider the following:
	\begin{gather*}
	x \in S_{ij}
	\iff \abs{x-a_i} < \abs{x-a_j}
	\iff \abs{x-a_j}^2 - \abs{x-a_i}^2 > 0 \\
	\iff \dpair{x-a_j,x-a_j} + \dpair{x-a_i,a_i-x} > 0 \\
	\iff \dpair{x-a_j,x-a_j} + \dpair{x-a_i,a_i-x}
	   + \dpair{x-a_j,a_i-x} + \dpair{x-a_i,x-a_j} > 0 \\
	\iff \dpair{x-a_j,x-a_j} + \dpair{x-a_j,a_i-x}
	   + \dpair{x-a_i,a_i-x} + \dpair{x-a_i,x-a_j} > 0 \\
	\iff \dpair{x-a_j,x-a_j+a_i-x} + \dpair{x-a_i,a_i-x+x-a_j} > 0 \\
	\iff \dpair{x-a_j+x-a_i,a_i-a_j} > 0
	\iff \dpair{x-\frac{a_i+a_j}{2},a_i-a_j} > 0.
	\end{gather*}
	We can now recognize the definition of open half-plane, the midpoint
	$(a_i+a_j)/2$ and the direction $a_i-a_j$ of the line segment $a_i a_j$.
	This proves that $\tilde{\Omega}_i$ is convex, because we've expressed
	$\tilde{\Omega}_i$ as the intersection of convex sets
	(the open semiplanes $S_{ij}$).
	Moreover, the intersection of open half-planes is a (topologically)
	open polygonal region, so its boundary must be a polygonal chain
	(allowing half-lines as edges, if $\tilde{\Omega}_i$ is unbounded).
\item[b)] Let $n = \#(\partial B(p) \cap \mathcal{A})$. To prove i),
	let $p$ be a Voronoi vertex. This means that $p$ is a vertex of
	the boundary of some Voronoi polygon $\tilde{\Omega}_i$, so
	there exist $j,k \in \bar{\Lambda}$ such that $i,j,k$ are all distinct
	and $p \in \partial S_{ij} \cap \partial S_{ik}$.
	The point $p$ is therefore the circumcenter of the triangle $a_i a_j a_k$,
	since we have proved in point a) that $\partial S_{ij}$ and $\partial S_{ik}$
	are the perpendicular bisectors of the sides $a_i a_j$ and $a_i a_k$.
	As Euclid could confirm, this in turn implies that $p$ has equal
	distance from all the vertices $a_i, a_j, a_k$ of the triangle, so $n \geq 3$
	because we know that $\{a_i,a_j,a_k\} \subseteq \partial B(p) \cap \mathcal{A}$.
	To prove the converse, we can follow the same line of reasoning.
	Let $n \geq 3$. Then there exist distinct
	$\{a_i,a_j,a_k\}$ in $\partial B(p) \cap \mathcal{A}$, and once again
	$p$ is the circumcenter of the triangle $a_i a_j a_k$.
	This means that $p \in \partial S_{ij} \cap \partial S_{ik}$,
	so $p$ is a Voronoi vertex.
	
	To prove ii), let $p$ be a point on the edge of a Voronoi polygon,
	say $\tilde{\Omega}_i$. We want to prove that $n=2$.
	On the one hand, we know that $n$ can't be greater than $2$, otherwise $p$
	would be a vertex by i) (we implicitly assume that $p$
	is not on the boundary of the edge).
	On the other hand, we know that $p \in \partial S_{ij}$ for some $j \neq i$,
	so $p$ must be the midpoint of $a_i a_j$.
	This means that, if $\partial B(p) \cap \mathcal{A}$ contains $a_i$
	(which it does, by the choice of $i$), then it must also	contain $a_j$.
	So $n$ can only be 2. To prove the converse, let $p \in \R^2$ such that $n = 2$.
	For the sake of contradiction, suppose that $p$ belongs to any of
	the open sets $\tilde{\Omega}_i$. Then $\abs{p-a_i} < \abs{p-a_j}$
	for every $j \neq i$, so $\partial B(p)$ can only contain
	one point from $\mathcal{A}$, namely $a_i$.
	This contradicts the assumption $n = 2$, so
	$p$ must be on the boundary of some Voronoi polygon.
	But $p$ cannot be a vertex, otherwise we would have $n \geq 3$
	by point i). Therefore $p$ can only be on an edge, as required. $\square$
\end{description}

\end{document}





























