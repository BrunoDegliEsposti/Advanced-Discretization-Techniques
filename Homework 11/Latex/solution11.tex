\documentclass[a4paper]{article}
\usepackage[T1]{fontenc}
\usepackage[utf8]{inputenc}
\usepackage[english]{babel}
\usepackage{amsmath}
\usepackage{amsfonts}
\usepackage{amssymb}
\usepackage{enumitem}
\usepackage{resizegather} \addtolength{\jot}{4pt}
\usepackage{microtype}

\renewcommand{\vec}[1]{\mathbf{#1}}
\DeclareMathOperator*{\argmin}{\arg\!\min}
\DeclareMathOperator*{\argmax}{\arg\!\max}
\DeclareMathOperator{\sgn}{sgn}
\DeclareMathOperator{\diver}{div}
\DeclareMathOperator{\supp}{supp}
\newcommand{\dx}{\, dx \,}
\newcommand{\dy}{\, dy \,}
\newcommand{\dt}{\, dt \,}
\newcommand{\ds}{\, ds \,}
\newcommand{\dxdy}{\, dx \:\! dy \,}
\newcommand{\dvecx}{\, d\vec{x} \,}
\newcommand{\dsigma}{\, d\sigma \,}
\newcommand{\area}[1]{\left\lvert #1 \right\rvert}
\newcommand{\abs}[1]{\left\lvert #1 \right\rvert}
\newcommand{\seminorm}[1]{\left\lvert #1 \right\rvert}
\newcommand{\norm}[1]{\left\lVert #1 \right\rVert}
\newcommand{\dpair}[1]{\left\langle #1 \right\rangle}
\newcommand{\R}{\mathbb{R}}
\newcommand{\N}{\mathbb{N}}
\newcommand{\C}{\mathbb{C}}
\renewcommand{\Re}{\operatorname{Re}}
\renewcommand{\Im}{\operatorname{Im}}

\title{\huge{Advanced Discretization Techniques \\ Homework 11}}
\author{\Large{Bruno Degli Esposti, Xingyu Xu}}
\date{January 14th - January 21st, 2020}

\begin{document}

\maketitle

\section*{Exercise 20: Stability of the one-step $\theta$ method}
The one-step $\theta$ method is defined as the iterative scheme
\[
\xi^0 = \xi_0, \quad
\frac{\xi^{n+1}-\xi^n}{\Delta t} + \lambda(\theta\xi^{n+1}+(1-\theta)\xi^n) = 0,
\]
which can be reformulated as
\[
\xi^0 = \xi_0, \quad
\xi^{n+1} = \frac{1-\lambda\Delta t(1-\theta)}{1+\lambda\Delta t\theta}\xi^n.
\]
Therefore, the complex function $R_\theta$ is defined as
\[
R_\theta(z)=\frac{1+(1-\theta)z}{1-\theta z},
\]
so that $\xi^{n+1} = R_\theta(-\lambda \Delta t) \xi^n$.
If $\theta \in \{0,\frac{1}{2},1\}$, then we have
\[
R_0(z) = z+1,
\qquad R_{1/2}(z) = \frac{1+z/2}{1-z/2},
\qquad R_1(z) = \frac{1}{1-z}.
\]
Now we can determine the three domains of stability:
\begin{itemize}
\item $S_{R_0}$ is a circle with radius $1$ and center $(-1,0)$:
\begin{align*}
S_{R_0}
&	= \{ z \in \C \mid \abs{z+1} < 1 \} \\
&	= \{ z \in \C \mid (z+1)\overline{(z+1)} < 1 \} \\
&	= \{ x+yi \in \C \mid (x+yi+1)(x-yi+1) < 1 \} \\
&	= \{ x+yi \in \C \mid (x+1)^2 + y^2 < 1 \}.
\end{align*}
\item $S_{R_{1/2}}$ is the left half-plane:
\begin{align*}
S_{R_{1/2}}
&	= \left\{ z \in \C \mid \abs{\frac{1+z/2}{1-z/2}} < 1 \right\} \\
&	= \left\{ z \in \C \mid \left(\frac{1+z/2}{1-z/2}\right)
                            \left(\frac{1+\bar{z}/2}{1-\bar{z}/2}\right) < 1 \right\} \\
&	= \left\{ z \in \C \mid \frac{1+\Re(z)+z\bar{z}/4}{1-\Re(z)+z\bar{z}/4} < 1 \right\} \\
&	= \left\{ z \in \C \mid 1+\Re(z)+z\bar{z}/4 < 1-\Re(z)+z\bar{z}/4 \right\} \\
&	= \left\{ z \in \C \mid 2\Re(z) < 0 \right\}.
\end{align*}
\item $S_{R_1}$ is the whole complex plane, minus the unit circle centered at $(1,0)$:
\begin{align*}
S_{R_1}
&	= \{ z \in \C \mid \abs{\frac{1}{1-z}} < 1 \} \\
&	= \{ z \in \C \mid \abs{1-z} > 1 \} \\
&	= \{ z \in \C \mid (1-z)\overline{(1-z)} > 1 \} \\
&	= \{ x+yi \in \C \mid (1-x-yi)(1-x+yi) > 1 \} \\
&	= \{ x+yi \in \C \mid (1-x)^2 + y^2 > 1 \}.
\end{align*}
\end{itemize}
Moving on to L-stability, we have that
\begin{align*}
\lim_{Re(z)\to-\infty} R_\theta(z)
&	= \lim_{Re(z)\to-\infty} \frac{1+(1-\theta)z}{1-\theta z} \\
&	= \lim_{Re(z)\to-\infty} \frac{1/z+(1-\theta)}{1/z-\theta}
	= \frac{1-\theta}{-\theta}
	= 0
\end{align*}
if and only if $\theta = 1$ (if $\theta = 0$, the limit doesn't even exist). $\square$

\section*{Exercise 21: Fully discrete estimate}
The implicit Euler method is defined as the iterative scheme
\[
U^0 = u_0, \quad
M \frac{U^{n+1}-U^n}{\tau_n} + AU^{n+1} = F^{n+1}.
\]
We begin the proof by testing the last equation with $U^{n+1}$:
\[
(U^{n+1})^T M \frac{U^{n+1}-U^n}{\tau_n} + (U^{n+1})^TAU^{n+1} = (U^{n+1})^TF^{n+1}.
\]
At the continuous level, this identity corresponds to
\begin{equation}
\int_\Omega u^{n+1} \left( \frac{u^{n+1}-u^n}{\tau_n} \right) \dx
+ \int_\Omega \nabla u^{n+1} \cdot \nabla u^{n+1} \dx
= \int_\Omega f^{n+1} u^{n+1} \dx
\end{equation}
Using (1) and Cauchy-Schwarz's inequality (twice),
we can now prove that the following estimate holds for each timestep $\tau_n$:
\addtolength{\jot}{6pt}
\begin{align*}
\frac{\|u^{n+1}\|_0-\|u^{n}\|_0}{\tau_n}
&	=\frac{\|u^{n+1}\|^2_0-\|u^n\|_0\|u^{n+1}\|_0}{\tau_n\|u^{n+1}\|_0}
	\leq\frac{\|u^{n+1}\|^2_0-\langle u^n,u^{n+1}\rangle_0}{\tau_n\|u^{n+1}\|_0} \\
&	=\frac{1}{\|u^{n+1}\|_0}\int_\Omega u^{n+1}\left(\frac{u^{n+1}-u^n}{\tau_n}\right)\dx \\
&	\stackrel{(1)}{=}\frac{-\seminorm{u^{n+1}}_1
		+\langle f^{n+1},u^{n+1}\rangle_0}{\|u^{n+1}\|_0} \\
&	\leq\frac{\|f^{n+1}\|_0\|u^{n+1}\|_0}{\|u^{n+1}\|_0}
	=\|f^{n+1}\|_0,
\end{align*}
which is equivalent to
\[
\norm{u^{n+1}}_0 - \norm{u^n}_0
\leq \tau_n \norm{f^{n+1}}_0.
\]
Taking the sum on both sides as $n$ goes from $0$ to $s-1$, we finally obtain
\[
\sum_{n=0}^{s-1} \left( \norm{u^{n+1}}_0 - \norm{u^n}_0 \right)
= \norm{u^s}_0 - \norm{u^0}_0
\leq \sum_{n=0}^{s-1} \tau_n \norm{f^{n+1}}_0,
\]
which is a fully-discrete equivalent to the estimate proven in Exercise 19a). $\square$
\end{document}





