\documentclass[a4paper]{article}
\usepackage[T1]{fontenc}
\usepackage[utf8]{inputenc}
\usepackage[english]{babel}
\usepackage{amsmath}
\usepackage{amsfonts}
\usepackage{amssymb}
\usepackage{stmaryrd}
\usepackage{resizegather} \addtolength{\jot}{4pt}
\usepackage{microtype}

\renewcommand{\vec}[1]{\mathbf{#1}}
\DeclareMathOperator{\diver}{div}
\DeclareMathOperator{\supp}{supp}
\newcommand{\abs}[1]{\left\lvert#1\right\rvert}
\newcommand{\dx}{\, dx \,}
\newcommand{\dy}{\, dy \,}
\newcommand{\dgamma}{\, d\gamma}
\newcommand{\dsigma}{\, d\sigma}
\newcommand{\loneloc}{L^1_{\text{loc}}(\Omega)}
\newcommand{\norm}[1]{\left\lVert#1\right\rVert}
\newcommand{\norminf}[1]{\left\lVert#1\right\rVert_\infty}
\newcommand{\normtwo}[1]{\left\lVert#1\right\rVert_2}
\newcommand{\normltwo}[1]{\left\lVert#1\right\rVert_{L^2}}
\newcommand{\normhone}[1]{\left\lVert#1\right\rVert_{H^1}}
\newcommand{\R}{\mathbb{R}}
\newcommand{\N}{\mathbb{N}}
\newcommand{\Pone}{\mathbb{P}_1}
\newcommand{\seminorm}[1]{\left\lvert#1\right\rvert}

\title{\huge{Advanced Discretization Techniques \\
Homework 2}}
\author{\Large{Bruno Degli Esposti}}
\date{October 29th - November 5th, 2019}

\begin{document}

\maketitle

\section*{Exercise 5: A different kind of element}
\begin{description}
\item[$a)$] Let $\{e_1,\dots,e_d\}$ be the canonical basis of $\R^d$.
	We will first prove the statement for the case $T = \hat{T}$,
	where $\hat{T}$ is the convex hull of the points $\{0,e_1,\dots,e_d\}$.
	The volume of $\hat{T}$ is $1/d!$. The functions
	\[
	v \mapsto \int_T v(y) \dy \qquad \text{and} \qquad
	v \mapsto \frac{1}{d!} v\left(\frac{1}{d+1},\dots,\frac{1}{d+1}\right)
	\]
	belong to $\Pone^*$, so in order to show that they are equal it suffices to check
	that they agree on a basis of $\Pone$. % \subset \R[y_1,\dots,y_d]$.
	In this proof, we will choose the monomial basis.
	The case $v(y) = 1$ is trivial. If $v(y) = y_1$, then
	\begin{gather*}
	\int_T v(y) \dy
	= \int_T y_1 \, dy_1 \dots dy_d
	= \int_0^1 \left( \int_{L(y_1)} y_1 \, dy_2 \dots dy_d \right) \, dy_1
	= \int_0^1 y_1 \abs{L(y_1)} \, dy_1
	\end{gather*}
	by Fubini's theorem, where
	\[
	L(y_1) = \{y_1\} \times \{(y_2,\dots,y_d) \in \R_+^{d-1}
	\mid y_2+\dots+y_d \leq 1 - y_1\}.
	\]
	Now,
	\[
	\abs{L(y_1)} = \abs{(1-y_1)L(0)}
	= (1-y_1)^{d-1} \abs{L(0)} = \frac{(1-y_1)^{d-1}}{(d-1)!},
	\]
	hence
	\begin{gather*}
	\int_0^1 y_1 \abs{L(y_1)} \, dy_1
	= \frac{1}{(d-1)!} \int_0^1 y_1 (1-y_1)^{d-1} \, dy_1 \\
	= 0 - \frac{1}{(d-1)!} \int_0^1 -\frac{(1-y_1)^d}{d} \, dy_1 
	= \frac{1}{(d-1)!} \left[ -\frac{(1-y_1)^{d+1}}{d(d+1)} \right]_0^1 \\
	= \frac{1}{(d+1)!}
	= \frac{1}{d!} v\left(\frac{1}{d+1},\dots,\frac{1}{d+1}\right),
	\end{gather*}
	as was requested. The cases $v(y) = y_2,\dots,v(y) = y_d$
	can be proven in the same way. Now we prove the theorem for a generic $T$.
	Let $m \colon \hat{T} \mapsto T$ be the affine mapping sending $\hat{T}$ to $T$.
	Then
	\begin{gather*}
	\int_T v(y) \dy
	= \int_{\hat{T}} v(m(y)) \abs{\det(J_m)} \dy
	= \abs{\det(J_m)} \int_{\hat{T}} v(m(y)) \dy \\
	= d!\abs{T} \frac{1}{d!} \,
		v\left(m\left(\frac{1}{d+1},\dots,\frac{1}{d+1}\right)\right)
	= \abs{T} \, v\left( \frac{1}{d+1} \sum_{i=0}^d x_i \right)
	\end{gather*}
\item[$b)$] The only nontrivial condition to check is that the functions
	$\{\sigma_0,\dots,\sigma_d\}$ are linearly independent.
	Let $a_0,\dots,a_d \in \R$ be such that
	\[
	a_0 \sigma_0 + \dots + a_d \sigma_d = 0.
	\]
	We want to show that all $a_i$ are 0. As per the hint, we can evaluate the
	linear combination in $\theta_i(y) = 1 - d \lambda_i(y) \in \Pone$:
	\begin{gather*}
	0 = \sum_{j=0}^d a_j \sigma_j(\theta_i)
	= \sum_{j=0}^d \frac{a_j}{\abs{e_j}} \int_{e_j} \theta_i(y) \dy
	\stackrel{a)}{=} \sum_{j=0}^d \frac{a_j}{\abs{e_j}} \abs{e_j}
		\theta_i \left( \frac{1}{d} \sum_{k=0,k \neq j}^d x_k \right) \\
	= \sum_{j=0}^d a_j \left(
		1 - d \lambda_i \left(
			\frac{1}{d} \sum_{k=0,k \neq j}^d x_k
		\right)
	\right)
	= \sum_{j=0}^d a_j \delta_{ij}
	= a_i
	\end{gather*}
	for all $i = 0,\dots,d$, as required for linear independence.
\item[$c)$] Let $p_1$ and $p_2$ be the restrictions of $v_h$ to $T_1$ and $T_2$,
	two elements sharing the face $e$. Let $e_c$ be the centroid of $e$.
	Since $v_h$ is piecewise affine,	the size of the jump discontinuity
	$\llbracket v_h \rrbracket$ across $e$ is given by $p_1 - p_2$, and
	$v_h$ is continuous in $e_c$ if and only if $p_1(e_c) = p_2(e_c)$.
	Let's prove this equality:
	\begin{gather*}
	0 = \int_e \llbracket v_h \rrbracket \dy
	= \int_e p_1(y) - p_2(y) \dy \\
	\int_e p_1(y) \dy = \int_e p_2(y) \dy
	\stackrel{a)}{\Rightarrow} \abs{e} p_1(e_c) = \abs{e} p_2(e_c)
	\Rightarrow p_1(e_c) = p_2(e_c). \quad \square
	\end{gather*}
\end{description}

\section*{Exercise 6: A different kind of elliptic equation}
\begin{description}
\item[$a)$] We multiply the equation by a test function
	$\varphi \in X = H^2_0(\Omega)$, then integrate over $\Omega$,
	integrate by parts twice and use the fact that both $\varphi$
	and 	$\nabla \varphi$ vanish on the boundary of the domain:
	\begin{gather*}
	\int_\Omega \Delta^2 u \, \varphi \dx
	= \int_\Omega f \varphi \dx \\
	\int_{\partial \Omega} (\nabla (\Delta u) \cdot \vec{n}) \, \varphi \dsigma
	-\int_\Omega \nabla (\Delta u) \cdot \nabla \varphi \dx
	= \int_\Omega f \varphi \dx \\
	-\int_{\partial \Omega} \Delta u \, (\nabla \varphi \cdot \vec{n}) \dsigma
	+\int_\Omega \Delta u \, \Delta \varphi \dx
	= \int_\Omega f \varphi \dx \\
	\int_\Omega \Delta u \, \Delta \varphi \dx
	= \int_\Omega f \varphi \dx. \\
	\end{gather*}
	Let
	\[
	a(u,\varphi) = \int_\Omega \Delta u \, \Delta \varphi \dx,
	\quad f(\varphi) = \int_\Omega f \varphi \dx.
	\]
	Then the weak form of the equation is $a(u,\varphi) = f(\varphi)$.
	In the continuous setting, the problem is:
	\[
	\text{Find } u \in X \text{ such that }
	a(u,\varphi) = f(\varphi) \text{ holds for every } \varphi \in X.
	\]
	The system is not overconstrained because the weak gradient of every
	function in $H_0^2$ vanishes on the boundary (has trace 0),
	so the boundary condition $Du = 0$ is quite natural.
	Otherwise, by just asking for $u=0$ we would have to work
	in $H_0^1(\Omega) \cap H^2(\Omega)$, but then we can't integrate
	by parts a second time without introducing an unwanted extra term.
\item[$b)$] The first half of the proof boils down to Schwarz's theorem
	and integration by parts. Indeed, for each $\varphi \in C_0^\infty(\Omega)$,
	we have that
	\begin{gather*}
	\int_\Omega (\Delta\varphi(x))^2 \dx
	= \int_\Omega \big( \sum_{i=1}^d \partial_{x_i x_i} \varphi(x) \big)^2 \dx
	= \int_\Omega \sum_{i,j=1,\dots,d} \partial_{x_i x_i} \varphi(x) \,
	                                   \partial_{x_j x_j} \varphi(x) \dx \\
	= \int_\Omega \sum_{i,j=1,\dots,d} \partial_{x_i} \varphi(x) \,
	                                   \partial_{x_j x_j x_i} \varphi(x) \dx
	= \int_\Omega \sum_{i,j=1,\dots,d} \partial_{x_i} \varphi(x) \,
	                                   \partial_{x_i x_j x_j} \varphi(x) \dx \\
	= \int_\Omega \sum_{i,j=1,\dots,d} \partial_{x_i x_j} \varphi(x) \,
	                                   \partial_{x_i x_j} \varphi(x) \dx
	= \int_\Omega \sum_{\abs{\alpha}=2} \big( \partial_{x^\alpha} \varphi(x) \big)^2 \dx
	= \seminorm{\varphi}_2^2.
	\end{gather*}
	The second half of the proof follows by the density of
	$\varphi \in C_0^\infty(\Omega)$ in $H_0^2(\Omega)$ and the continuity
	of the two functionals, which is readily checked:
	\begin{align*}
	\int_\Omega (\Delta\varphi(x))^2 \dx
&	= \int_\Omega \big( \sum_{i=1}^d \partial_{x_i x_i} \varphi(x) \big)^2 \dx \\
&	\leq d \int_\Omega \sum_{i=1}^d \big( \partial_{x_i x_i} \varphi(x) \big)^2 \dx
	\leq d \seminorm{\varphi}_2^2
	\leq d \norm{\varphi}_2^2.
	\end{align*}
	The first inequality is just the usual AM-QM inequality.
\item[$c)$] We derive the following inequality, which will be useful in point d):
	\[
	\exists C > 0 \text{ such that }
	\seminorm{u}_2^2 \geq C \norm{u}_2^2 \text{ for each } u \in H_0^2(\Omega).
	\]
	The proof relies on the fact that weak derivatives of functions in $H_0^2(\Omega)$
	belong to $H_0^1(\Omega)$, so the basic version of Poincaré's inequality
	still applies to them: there exist constants $c,c_1,\dots,c_d > 0$ such that
	\[
	\normltwo{u}^2 \leq c \seminorm{u}_1^2 \text{ and }
	\normltwo{\partial_{x_i} u}^2 \leq c_i \seminorm{\partial_{x_i} u}_1^2
	\text{ for each } i = 1,\dots,d.
	\]
	From now on, it's just algebra:
	\begin{gather*}
	\normltwo{u}^2
	\leq c \seminorm{u}_1^2
	= c \normltwo{\nabla u}^2
	= c \sum_{i=1}^d \normltwo{\partial_{x_i} u}^2
	\leq c \sum_{i=1}^d c_i \seminorm{\partial_{x_i} u}_1^2 \\
	= c \sum_{i=1}^d c_i \normltwo{\nabla \partial_{x_i} u}^2
	\leq c' \sum_{i,j=1}^d \normltwo{\partial_{x_i x_j} u}^2
	= c' \seminorm{u}_2^2.
	\end{gather*}
	Comparing the first, second and last terms we get that
	\[
	\norm{u}_2^2
	= \normltwo{u}^2 + \seminorm{u}_1^2 + \seminorm{u}_2^2
	\leq \left( c' + \frac{c'}{c} + 1 \right) \seminorm{u}_2^2
	= c'' \seminorm{u}_2^2,
	\]
	so we can just choose $C = 1/c''$.
\item[$d)$] Existence and uniqueness of a weak solution $u \in X$ is given by
	the Lax-Milgram theorem applied to the equation $a(u,v)=f(v)$.
	Let's check that the hypothesis are satisfied.
	The continuity of $f$ is given by $f \in X^*$.
	The continuity of $a(\cdot,\cdot)$ follows from the Cauchy-Schwarz inequality
	and point~b):
	\begin{gather*}
	a(u,v)
	= \int_\Omega \Delta u \, \Delta v \dx
	\leq \left( \int_\Omega (\Delta u)^2 \dx \right)^{1/2}
	     \left( \int_\Omega (\Delta v)^2 \dx \right)^{1/2} \\
	= \seminorm{u}_2 \seminorm{v}_2
	\leq \norm{u}_2 \norm{v}_2.
	\end{gather*}
	The coercivity of $a(\cdot,\cdot)$ follows from points b) and c):
	\begin{gather*}
	a(u,u) = \int_\Omega (\Delta u)^2 \dx = \seminorm{u}_2^2 \geq C \norm{u}_2^2. \quad \square
	\end{gather*}
\end{description}

\end{document}





























