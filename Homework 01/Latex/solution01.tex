\documentclass[a4paper]{article}
\usepackage[T1]{fontenc}
\usepackage[utf8]{inputenc}
\usepackage[english]{babel}
\usepackage{amsmath}
\usepackage{amsfonts}
\usepackage{amssymb}
\usepackage{resizegather} \addtolength{\jot}{4pt}
\usepackage{microtype}

\renewcommand{\vec}[1]{\mathbf{#1}}
\DeclareMathOperator{\diver}{div}
\DeclareMathOperator{\supp}{supp}
\newcommand{\abs}[1]{\left\lvert#1\right\rvert}
\newcommand{\dx}{\, dx \,}
\newcommand{\dgamma}{\, d\gamma}
\newcommand{\dsigma}{\, d\sigma}
\newcommand{\loneloc}{L^1_{\text{loc}}(\Omega)}
\newcommand{\norm}[1]{\left\lVert#1\right\rVert}
\newcommand{\norminf}[1]{\left\lVert#1\right\rVert_\infty}
\newcommand{\normtwo}[1]{\left\lVert#1\right\rVert_2}
\newcommand{\normltwo}[1]{\left\lVert#1\right\rVert_{L^2}}
\newcommand{\normhone}[1]{\left\lVert#1\right\rVert_{H^1}}
\newcommand{\R}{\mathbb{R}}
\newcommand{\N}{\mathbb{N}}
\newcommand{\seminorm}[1]{\left\lvert#1\right\rvert}

\title{\huge{Advanced Discretization Techniques \\
Homework 1}}
\author{\Large{Bruno Degli Esposti}}
\date{22-29 October 2019}

\begin{document}

\maketitle

\section*{Exercise 1: Minimization property}
\begin{description}
\item[$\Rightarrow)$]
Let $u \in U$ be a solution of $J(u) = \inf_{v \in U} J(v)$.
For each $v \in U$, the segment $\{u + t(v-u) \mid t \in [0,1]\}$
belongs to $U$ (by the convexity assumption).
Therefore, for all $t \in (0,1)$, we have
\begin{gather*}
J(u) \leq J(u + t(v-u)) \\
\frac{1}{2}a(u,u) - f(u) \leq \frac{1}{2}a(u+t(v-u),u+t(v-u)) - f(u+t(v-u)) \\
0 \leq t a(u,v-u) + \frac{1}{2} t^2 a(v-u,v-u) - t f(v-u) \\
a(u,v-u) - f(v-u) \geq -\frac{1}{2} t a(v-u,v-u)
\end{gather*}
By taking the limit as $t \to 0$, we get the required inequality.
\item[$\Leftarrow)$]
Let $u$ be an element of $U$ satisfying
\[
a(u,v-u) \geq f(v-u) \quad \text{for all $v \in U$.}
\]
For a fixed $v \in U$, we have to show that $J(u) \leq J(v)$.
Since $a$ is coercive, $a(v-u,v-u) \geq 0$.
Then, by algebraic manipulations exploiting linearity, bilinearity and simmetry
of $f$ and $a$, we get
\begin{gather*}
a(u,v-u) + \frac{1}{2} a(v-u,v-u) \geq f(v-u) \\
\frac{1}{2} a(u+(v-u),u+(v-u)) - \frac{1}{2} a(u,u) \geq f(v) - f(u) \\
\frac{1}{2} a(v,v) - f(v) \geq \frac{1}{2} a(u,u) - f(u) \\
J(v) \geq J(u). \quad \square
\end{gather*}
\end{description}

\section*{Exercise 2: Chain rule for weak derivatives}
% https://www.desmos.com/calculator/fe049txwkn
\begin{description}
\item[$a1)$] Since the proof is quite long, we've split it up over 5 points.
	We begin with the remark that, by Lagrange's theorem,
	$f$ is a Lipschitz function with constant $L = \norminf{f'}$,
	hence $\abs{f(x)} \leq L \abs{x} + \abs{f(0)}$ for all $x \in \R$.
\item[$a2)$] Let $g = f \circ u$ and $h_i = f'(u) \partial_{x_i} u$
	for each $i = 1,\dots,d$. Let $\mu$ be the Lebesgue measure on $\R^d$.
	Before checking that the definition of weak derivative
	holds, we first have to prove that $g$ and $\{h_i\}_{i=1,\dots,d}$ all
	belong to $\loneloc$.
	For each compact subset $K$ of $\Omega$ we have that
	\begin{multline*}
	\int_K \abs{g(x)} \dx
	= \int_K \abs{f(u(x))} \dx
	\leq \int_K L \abs{u(x)} + \abs{f(0)} \dx \\
	= L \int_K \abs{u(x)} \dx + \abs{f(0)} \mu(K)
	< +\infty,
	\end{multline*}
	because $u \in \loneloc$ and $K$ is bounded. We also have that
	\begin{equation*}
	\int_K \abs{h_i(x)} \dx
	= \int_K \abs{f'(u(x)) \partial_{x_i}u} \dx
	\leq L \int_K \abs{\partial_{x_i}u} \dx
	< +\infty
	\end{equation*}
	for all $i = 1,\dots,d$, because $\partial_{x_i}u \in \loneloc$.
\item[$a3)$] Since $C^\infty(\Omega)$ is a dense subset of
	$W^{1,1}_\text{loc}(\Omega)$, we can find a sequence
	$\{u_m\}$ in $C^\infty(\Omega)$ such that $u_m \to u$.
	Explicitly, this means that $\norm{u - u_m}_{W^{1,1}} \to 0$
	on every compact subset of $\Omega$.
	Without loss of generality, we can even ask for $\{u_m\}$ to be
	a.e.\ pointwise convergent to $u$ (just extract an appropriate subsequence).
	This additional property will be useful later in order to apply
	Lebesgue's dominated convergence theorem.
	At this point we would like to apply the theorem given in the hint,
	but the problem is that $f \circ u_m$ is only a $C^1$ function,
	not $C^\infty$. Therefore out proof will not rely on it.
\item[$a4)$] We can now check that the definition of weak derivative holds:
	\begin{gather*}
	\text{For all $i = 1,\dots,d$ and $\varphi \in C^\infty_0(\Omega)$,} \quad
	\int_\Omega g(x) \partial_{x_i} \varphi(x) \dx \\
	= \int_{\supp(\varphi)} (f(u(x)) - f(u_m(x))) \partial_{x_i} \varphi(x) \dx 
	+ \int_{\supp(\varphi)} f(u_m(x)) \partial_{x_i} \varphi(x) \dx \\
	= (1) - \int_{\supp(\varphi)} f'(u_m(x)) \partial_{x_i} u_m(x) \varphi(x) \dx \\
	= (1) - \int_{\supp(\varphi)} (f'(u_m(x))\partial_{x_i}u_m(x) - h_i(x)) \varphi(x) \dx
	      - \int_{\supp(\varphi)} h_i(x) \varphi(x) \dx \\
	= (1) - (2) - \int_\Omega h_i(x) \varphi(x) \dx
	\end{gather*}
	All that's left to do is to prove that the integrals $(1)$ and $(2)$
	vanish in the limit as $m \to \infty$.
\item[$a5)$] Let's check the first integral:
	\begin{gather*}
	\lim_{m \to \infty} \abs{(1)}
	\leq \lim_{m \to \infty} \int_{\supp(\varphi)} \abs{f(u(x)) - f(u_m(x))}
	                         \abs{\partial_{x_i}\varphi(x)} \dx \\
	\leq \norminf{\partial_{x_i}\varphi(x)} L
	     \lim_{m \to \infty} \int_{\supp(\varphi)} \abs{u(x) - u_m(x)} \dx
	= 0 \quad \text{by a3),}
	\end{gather*}
	because $\supp(\varphi)$ is a compact set and
	$\norm{\cdot}_{L^1} \leq \norm{\cdot}_{W^{1,1}}$.
	
	Now we check the second one:
	\begin{align*}
	\lim_{m \to \infty} \abs{(2)}
	& \leq \lim_{m \to \infty} \int_{\supp(\varphi)}
	       \abs{(f'(u_m(x))\partial_{x_i}u_m(x) - f'(u(x))\partial_{x_i}u(x))}
	       \abs{\varphi(x)} \dx \\
	& \leq \lim_{m \to \infty} \int_{\supp(\varphi)}
	       \abs{(f'(u_m(x))\partial_{x_i}u_m(x) - f'(u_m(x))\partial_{x_i}u(x))}
	       \abs{\varphi(x)} \dx \\
	& +    \lim_{m \to \infty} \int_{\supp(\varphi)}
           \abs{(f'(u_m(x))\partial_{x_i}u(x) - f'(u(x))\partial_{x_i}u(x))}
	       \abs{\varphi(x)} \dx \\
	& \leq \norminf{\varphi(x)} L \lim_{m \to \infty} \int_{\supp(\varphi)}
	       \abs{\partial_{x_i}u_m(x) - \partial_{x_i}u(x)} \dx \\
	& +    \norminf{\varphi(x)} \lim_{m \to \infty} \int_{\supp(\varphi)}
           \abs{(f'(u_m(x))\partial_{x_i}u(x) - f'(u(x))\partial_{x_i}u(x))} \dx \\
    & = 0 + 0 = 0.
	\end{align*}
	The first limit goes to zero by a3) again.
	The second limit goes to zero by the dominated convergence theorem,
	whose hypothesis are satisfied by the a.e.\ pointwise convergence of $\{u_m\}$,
	the continuity of $f'$ and the fact that $L \, \partial_{x_i} u(x)$ is integrable
	on every compact set and dominates the sequence $\{f'(u_m(x))\partial_{x_i}u(x)\}$.
	This completes the proof.

\item[$b)$] The main idea behind this proof (as per the hint) is quite simple,
	but making it rigorous requires a bit of effort.
	First, we prove some properties about the $f_\varepsilon$ family of functions.
	For each $\varepsilon > 0$, $f_\varepsilon$ is in $C^1(\R)$ because
	\[
	\lim_{u \to 0^+} f_\varepsilon(u)
	= (\varepsilon^2)^{1/2}-\varepsilon = 0, \qquad
	\lim_{u \to 0^+} f'_\varepsilon(u)
	= \lim_{u \to 0^+} \frac{u}{\sqrt{u^2+\varepsilon^2}} = 0.
	\]
	Moreover, $f'_\varepsilon(u)$ is in $L^\infty(\R)$:
	\[
	\abs{f'_\varepsilon(u)} = \abs{\frac{u}{\sqrt{u^2+\varepsilon^2}}}
	\leq \abs{\frac{u}{\sqrt{u^2}}} = 1
	\]
	
	Since $L^2(\Omega)$ is a subset of $L^1_\text{loc}(\Omega)$,
	both $u$ and $Du$ belong to $L^1_\text{loc}(\Omega)$ and therefore
	we can apply the theorem in a) to get that $f_\varepsilon(u(x))$
	has weak derivative $f'_\varepsilon(u(x)) Du(x)$.
	However, this is not enough to prove that $f_\varepsilon(u(x))$ is in $H^1$,
	because we still don't know if it belongs to $L^2$ or not (the same goes
	for the weak derivative). Let's see how to fix this.
	
	Let $f(u) = \max\{u,0\}$. As $\varepsilon$ goes monotonically to $0$
	($\varepsilon \searrow 0$), we have that $f_\varepsilon$ converges
	pointwise and monotonically to $f$ ($f_\varepsilon \nearrow f$),
	because the derivative of $f_\varepsilon(u)$ with respect
	to $\varepsilon$ is nonpositive for every fixed $u$.
	We also have that $f'_\varepsilon \nearrow \textbf{1}_{(0,+\infty)}$,
	the indicator function on $(0,+\infty)$.
	
	Next, we turn our attention to $u^+(x)$ and
	$v_i = \partial_{x_i} u(x) \cdot \textbf{1}_{(0,+\infty)}(u(x))$.
	All of these functions belong to $L^2(\Omega)$, because they are truncations
	of the $L^2$ functions $u$ and $\partial_{x_i} u$.
	Additionally, it follows from the monotone pointwise convergence that
	\[
	0 \leq f_\varepsilon(u(x)) \leq u^+(x), \quad
	0 \leq f'_\varepsilon(u(x)) \partial_{x_i} u(x)
	\leq \textbf{1}_{(0,+\infty)}(u(x)) \partial_{x_i} u(x)
	\]
	for every $x$ in $\Omega$ and $\varepsilon > 0$, and this is enough
	to prove that $f_\varepsilon(u(x))$ and
	$f'_\varepsilon(u(x)) \partial_{x_i} u(x)$ also belong to $L^2(\Omega)$.
	Therefore $f_\varepsilon(u(x)) \in H^1(\Omega)$ for each $\varepsilon > 0$,
	just like we wanted.
	
	Now we can move on to the second half of the proof, in which we try
	to approximate $u^+(x)$ with the sequence of $H^1$ functions $\{f_{1/n}(u(x))\}$.
	Giving a direct proof that $\{f_{1/n}(u(x))\}$ is a Cauchy sequence is not so easy.
	What we will do instead is prove that
	\[
	f_{1/n}(u(x)) \xrightarrow{L^2} u^+ \quad \text{and}	\quad
	f'_{1/n}(u(x)) \partial_{x_i} u(x) \xrightarrow{L^2}
	\textbf{1}_{(0,+\infty)}(u(x)) \partial_{x_i} u(x).
	\]
	From this, it follows that $f_{1/n}(u(x))$ and $f'_{1/n}(u(x)) \partial_{x_i} u(x)$
	are Cauchy sequences in $L^2$, and therefore $f_{1/n}(u(x))$
	must be a Cauchy sequence in $H^1$, as the following inequality shows:
	\begin{gather*}
	\normhone{ f_{1/n}(u(x))-f_{1/m}(u(x)) } \\
	= \left( \normltwo{ f_{1/n}(u(x))-f_{1/m}(u(x)) }^2
	+ \sum_{i=1}^d \normltwo{ f'_{1/n}(u(x)) \partial_{x_i} u(x)
	- f'_{1/m}(u(x)) \partial_{x_i} u(x)}^2 \right)^{1/2} \\
	\leq \normltwo{ f_{1/n}(u(x))-f_{1/m}(u(x)) }
	+ \sum_{i=1}^d \normltwo{ f'_{1/n}(u(x)) \partial_{x_i} u(x)
	- f'_{1/m}(u(x)) \partial_{x_i} u(x)}
	\end{gather*}
	The space $H^1$ is complete, so there is an element $w$ in $H^1$ to which
	the sequence $\{f_{1/n}(u(x))\}$ converges.
	Then, it must also converge in the $L^2$ topology,
	and by uniqueness of limits in $L^2(\Omega)$ (a metric space) we have that
	$w = u^+$, so we've proved that $u^+ \in H^1(\Omega)$ and
	\[
	f_{1/n}(u(x)) \xrightarrow{H^1} u^+(x),
	\]
	which in turn implies that 
	\[
	f'_{1/n}(u(x)) \partial_{x_i} u(x) \xrightarrow{L^2} \partial_{x_i}u^+(x)
	\quad \text{for each $i = 1,\dots,d$.}
	\]
	Again, by uniqueness of limits, we get
	$\partial_{x_i}u^+(x) = \textbf{1}_{(0,+\infty)}(u(x)) \partial_{x_i} u(x)$,
	as was required.
	Let us now check the aforementioned $L^2$ convergences:
	\begin{equation*}
	\lim_{n \to +\infty} \normltwo{f_{1/n}(u(x)) - u^+(x)}^2
	= \lim_{n \to +\infty} \int_\Omega \left( f_{1/n}(u(x)) - u^+(x) \right)^2 \dx
	= 0
	\end{equation*}
	by the dominated convergence theorem, which we can surely apply because
	$f_{1/n}(u) \nearrow u^+$ implies
	$\left( f_{1/n}(u(x)) - u^+(x) \right)^2 \searrow 0$
	(hence the integrand is bounded by $0$ and $\left( f_1(u(x)) - u^+(x) \right)^2$,
	both integrable functions on $\Omega$). This completes the proof for $u^+$.
	
	The proofs for $u^-$ and $\abs{u}$ now follow from the vector
	space structure of~$H^1$ and the linearity of the weak derivative, since
	\begin{gather*}
	u^-(x) = \min(u(x),0) = -\max(-u(x),0) = -(-u)^+(x) \\
	\abs{u(x)} = u^+(x) - u^-(x) = u^+(x) + (-u)^+(x). \quad \square
	\end{gather*}
\end{description}

\section*{Exercise 3: $H^1$-estimate for the Poisson problem}
\begin{description}
\item[$a)$] We multiply the Poisson equation by a test function
	$\varphi \in X = H^1_0(\Omega)$, then integrate over $\Omega$,
	integrate by parts and use the fact that $\varphi$
	vanishes on the boundary of the domain:
	\begin{gather*}
	- \int_\Omega \Delta u \varphi \dx
	= \int_\Omega f \varphi \dx \\
	- \int_{\partial\Omega} (\nabla u \cdot \vec{n})\varphi \dsigma
	+ \int_\Omega \nabla u \cdot \nabla \varphi \dx
	= \int_\Omega \nabla u \cdot \nabla \varphi \dx
	= \int_\Omega f \varphi \dx.
	\end{gather*}
	Let
	\[
	a(u,\varphi) = \int_\Omega \nabla u \cdot \nabla \varphi \dx,
	\quad f(\varphi) = \int_\Omega f \varphi \dx.
	\]
	Then the weak form of the Poisson equation is $a(u,\varphi) = f(\varphi)$.
	In the continuous setting, the problem is:
	\[
	\text{Find } u \in X \text{ such that }
	a(u,\varphi) = f(\varphi) \text{ holds for every } \varphi \in X.
	\]
	In the discrete setting, the problem is:
	\[
	\text{Find } U_h \in X_h \text{ such that }
	a(U_h,\varphi_h) = f(\varphi_h) \text{ holds for every } \varphi_h \in X_h.
	\]
\item[$b)$] Since $\varphi_h$ belongs to $X_h \subset X$, we have that
\begin{align*}
	\int_\Omega \nabla(U_h-u) \cdot \nabla\varphi_h \dx
	= a(U_h-u,\varphi_h)
&	= a(U_h,\varphi_h) - a(u,\varphi_h) \\
&	\stackrel{\text{a)}}{=} f(\varphi_h) - f(\varphi_h)
	= 0.
\end{align*}
\item[$c)$] We begin by introducing some notation:
	\begin{itemize}
	\item $\alpha$ is the coercivity constant of the bilinear form $a(\cdot,\cdot)$
	\item $M$ is its continuity constant
	\item $\mathcal{T}_h$ is a triangulation of $\Omega$ with maximum diameter $h$
	\item $a_1,\dots,a_M$ are the degrees of freedom of the triangulation
	\item $\varphi_1,\dots,\varphi_M$ is the nodal basis corresponding to the DOF
	\end{itemize}
	Cea's lemma asserts that
	\[
	\norm{u-U_h}_1 \leq \frac{M}{\alpha} \inf_{v_h \in X_h} \norm{u-v_h}_1.
	\]
	The Lagrange interpolation operator is defined as the map
	\[
	I_h \colon X \to X_h, \quad I_h(u) = \sum_{i = 1}^M u(a_i) \varphi_i.
	\]
	However, pointwise evaluations of $u$ on the degrees of freedom
	may not be well defined,
	because functions in $X = H^{k+1}(\Omega)$ are only defined
	up to a null set (technically, they are equivalence classes of usual functions).
	In order to avoid this problem, we require
	$H^{k+1}(\Omega) \subset C^0(\bar{\Omega})$, which is certainly the case
	if $k+1 > d/2$ (by one the many Sobolev embedding theorems).
	This explains the condition on $d$.
	Before proving the error estimate, we need one last inequality:
	\[
	\norm{u - I_h(u)}_1 \leq C h^k \seminorm{u}_{k+1}.
	\]
	Without going into too much detail, the inequality can be proven
	by splitting the norm into a sum over $\mathcal{T}_h$,
	then transforming the estimate on each triangle $K$
	to an estimate on a reference triangle $\hat{K}$
	with the help of an affine mapping $F$ (which must not become singular
	as h goes to 0; this is enforced by considering only regular triangulations),
	and finally making use of the Bramble–Hilbert lemma to (indirectly) estimate
	the interpolation error on every triangle by $h^k$ times
	the seminorm of $u$ on $K$ (ignoring constants).
	Now we can prove the required error estimate:
	\begin{align*}
	\norm{u-U_h}_1
&	\leq \frac{M}{\alpha} \inf_{v_h \in X_h} \norm{u-v_h}_1
	\leq \frac{M}{\alpha} \norm{u-I_h(u)}_1 \\
&	\leq \frac{M}{\alpha} C h^k \seminorm{u}_{k+1}
	\leq C' h^k \norm{u}_{k+1}. \quad \square
	\end{align*}
\end{description}

\section*{Exercise 4: $L^2$-estimate for the Poisson problem}
\begin{description}
\item[$a)$] Let $w \in X$ be the solution to the dual problem.
	Since $U_h - u$ also belongs to~$X$, by definition of dual problem we have that
	\[
	\norm{U_h-u}^2 = \langle U_h-u, U_h-u \rangle = a(w, U_h-u).
	\]
	The discrete solution $W_h$ is in $X_h$, so by Galerkin orthogonality
	\[
	a(w, U_h-u) = a(w, U_h-u) - a(U_h-u, W_h).
	\]
	The simmetry and linearity of $a(\cdot,\cdot)$ complete the proof.
\item[$b)$] For each $V_h \in X_h$, we have that
	\begin{align*}
	\norm{U_h-u}^2
&	= a(U_h-u, w-W_h)
	= a(U_h-u, w-W_h) + a(U_h-u, W_h-V_h) \\
&	= a(U_h-u, w-V_h)
	= \langle \nabla (U_h-u), \nabla (w-V_h) \rangle_{L^2(\Omega;\R^d)} \\
&	\leq \norm{\nabla (U_h-u)} \norm{\nabla (w-V_h)}, \text{ as required.}
	\end{align*}
	We have used Galerkin orthogonality and the Cauchy-Schwarz inequality.
\item[$c)$] Since $2 > 3/2 \geq d/2$,
	we get that $H^2(\Omega) \subset C^0(\bar{\Omega})$.
	So the Lagrange interpolation operator $I_h$ is well-defined
	on $w \in H^2(\Omega)$ and the following estimate holds:
	\[
	\norm{w-I_h(w)}_1 \leq C_1 h \norm{w}_2.
	\]
	Therefore
	\begin{align*}
	\inf_{V_h \in X_h} \norm{\nabla (w-V_h)}
&	\leq \norm{\nabla (w-I_h(w))}
	\leq \norm{w-I_h(w)}_1 \\
&	\leq C_1 h \norm{w}_2
	\leq C_2 h \norm{U_h-u}, \text{ as required.}
	\end{align*}
	In the last step we've used the elliptic regularity bound
	$\norm{w}_2 \leq C \norm{U_h-u}$.
\item[$d)$] Combining the results from 4b) and 4c) we get that
	\[
	\norm{U_h-u}^2 \leq C_2 h \norm{\nabla (U_h-u)} \norm{U_h-u}.
	\]
	Then we divide by $\norm{U_h-u}$ and use the result from 3c):
	\[
	\norm{U_h-u}
	\leq C_2 h \norm{\nabla (U_h-u)}
	\leq C_2 h \norm{U_h-u}_1
	\leq C_3 h^{k+1} \norm{u}_{k+1}. \quad \square
	\]
\end{description}

\end{document}





























